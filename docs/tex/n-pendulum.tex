\documentclass{article}


\usepackage{tikz}
\usepackage{times}
\usepackage{epsfig}
\usepackage{graphicx}
\usepackage{amsmath}
\usepackage{amssymb}
\usepackage{authblk}
\usepackage{physics} % for 'pdv' macro
\usepackage{booktabs}
\usepackage{fullpage}
\usepackage{siunitx}
\usepackage{caption}
\usepackage{subcaption}
\usepackage{pgfplots}
\pgfplotsset{compat=1.16}
\usepackage{comment}
\usepackage{array, makecell}
\usepackage{theorem,ifthen}
\usepackage{algorithm}
\usepackage[noend]{algpseudocode}
\usepackage{calrsfs}
\DeclareMathAlphabet{\pazocal}{OMS}{zplm}{m}{n}
%\newcommand{\La}{\mathcal{L}}
\newcommand{\N}{\pazocal{N}}
\newcommand{\E}{\pazocal{E}}

\newcommand{\bfA}{{\bf A}}
\newcommand{\bfB}{{\bf B}}
\newcommand{\bfC}{{\bf C}}
\newcommand{\bfD}{{\bf D}}
\newcommand{\bfE}{{\bf E}}
\newcommand{\bfF}{{\bf F}}
\newcommand{\bfG}{{\bf G}}
\newcommand{\bfH}{{\bf H}}
\newcommand{\bfI}{{\bf I}}
\newcommand{\bfJ}{{\bf J}}
\newcommand{\bfK}{{\bf K}}
\newcommand{\bfL}{{\bf L}}
\newcommand{\bfM}{{\bf M}}
\newcommand{\bfN}{{\bf N}}
\newcommand{\bfO}{{\bf O}}
\newcommand{\bfP}{{\bf P}}
\newcommand{\bfQ}{{\bf Q}}
\newcommand{\bfR}{{\bf R}}
\newcommand{\bfS}{{\bf S}}
\newcommand{\bfT}{{\bf T}}
\newcommand{\bfU}{{\bf U}}
\newcommand{\bfV}{{\bf V}}
\newcommand{\bfW}{{\bf W}}
\newcommand{\bfX}{{\bf X}}
\newcommand{\bfY}{{\bf Y}}
\newcommand{\bfZ}{{\bf Z}}

\newcommand{\bfa}{{\bf a}}
\newcommand{\bfb}{{\bf b}}
\newcommand{\bfc}{{\bf c}}
\newcommand{\bfe}{{\bf e}}
\newcommand{\bfg}{{\bf g}}
\newcommand{\bfh}{{\bf h}}
\newcommand{\bfj}{{\bf j}}
\newcommand{\bfk}{{\bf k}}
\newcommand{\bfi}{{\bf i}}
\newcommand{\bfs}{{\bf s}}
\newcommand{\bfx}{{\bf x}}
\newcommand{\bfy}{ {\bf y}}
\newcommand{\bfu}{{\bf u}}
\newcommand{\bfq}{{\bf q}}
\newcommand{\bfp}{{\bf p}}
\newcommand{\bfn}{{\bf n}}
\newcommand{\bfd}{{\bf d}}
\newcommand{\bfm}{{\bf m}}
\newcommand{\bfr}{{\bf r}}
\newcommand{\bff}{{\bf f}}
\newcommand{\bfv}{{\bf v}}
\newcommand{\bfw}{{\bf w}}
\newcommand{\bfz}{{\bf z}}
\newcommand{\bft}{{\bf t}}
\newcommand{\bfdr}{{\bf{dr}}}


\newcommand{\bfTheta}{{\boldsymbol \Theta}}
\newcommand{\bfPsi}{{\boldsymbol \Psi}}
\newcommand{\bflambda}{{\boldsymbol \lambda}}
\usepackage{calc}
\begin{document}


%%%%%%%%% TITLE
\title{n-pendulum}
\author[$\dagger$]{Tue Boesen}

\maketitle

We wish to simulate a n-pendulum, with massless rods.

In order to simulate this we use the Lagrangian approach, where we have a Lagrange function, $L$:
\begin{equation}
L = T - V,
\end{equation}
where $T$ is the kinetic energy, and $V$ is the potential energy.
\begin{equation}
V = g \sum_{i=1}^n m_i y_i, 
\end{equation}
\begin{equation}
T = \frac{1}{2} \sum_{i=1}^n m_i v_i^2,
\end{equation}
with $m$ being the mass of a pendulum, $g=-9.82 m/s^2$ being gravity acceleration and $v$ being the velocity.

The Lagrangian then has to obey the Euler-Lagrange equation:
\begin{equation}
\frac{d}{dt}\left( \frac{\partial L}{\partial \dot{q}_i} \right) = \frac{\partial L}{\partial q_i}
\end{equation}

We start by assuming all pendulums have a length of 1.
The cartesian coordinates and velocities are related to the angles and angular velocities in the following way:
\begin{equation}
x_i = \sum_{j=1}^i \sin(\theta_j)
\end{equation}
\begin{equation}
y_i = - \sum_{j=1}^i \cos (\theta_j)
\end{equation}
\begin{equation}
\dot{x}_i = \sum_{j=1}^i \dot{\theta}_j \cos(\theta_j)
\end{equation}
\begin{equation}
\dot{y}_i = \sum_{j=1}^i \dot{\theta}_j \sin(\theta_j)
\end{equation}

Hence we can express the kinetic energy as:
\begin{align}
T &= \frac{1}{2} \sum_{i} m_i v_i^2 \\ 
&= \frac{1}{2} \sum_i m_i \left( \left[ \sum_{j=1}^i \dot{\theta}_j \cos \theta_j \right]^2 + \left[ \sum_{j=1}^i  \dot{\theta}_j \sin \theta_j \right]^2 \right) \\
&= \frac{1}{2} \sum_i m_i \left( \sum_{j=1}^i \left[ \dot{\theta}_j^2 \cos^2 \theta_j + 2 \dot{\theta}_j \cos \theta_j \sum_{k = 1}^{j-1}  \dot{\theta}_k \cos\theta_k \right] + \sum_{j=1}^i \left[ \dot{\theta}_j^2 \sin^2 \theta_j + 2 \dot{\theta}_j \sin \theta_j \sum_{k =1}^{j-1}  \dot{\theta}_k \sin\theta_k  \right] \right) \\
&= \frac{1}{2} \sum_i m_i \left( \sum_{j=1}^i \dot{\theta}_j^2 + 2 \dot{\theta}_j \sum_{k =1 }^{j-1}  \dot{\theta}_k \cos \theta_j \cos\theta_k + 2 \dot{\theta}_j \sum_{k = 1}^{j-1}   \dot{\theta}_k \sin \theta_j  \sin\theta_k \right) \\
&= \frac{1}{2} \sum_i m_i \left( \sum_{j=1}^i \dot{\theta}_j^2 + 2 \dot{\theta}_j \sum_{k =1}^{j-1}  \dot{\theta}_k \left[\cos \theta_j \cos\theta_k + \sin \theta_j  \sin\theta_k \right] \right) \\
&= \frac{1}{2} \sum_i m_i \left( \sum_{j=1}^i \dot{\theta}_j^2 + 2 \dot{\theta}_j \sum_{k=1}^{j-1}  \dot{\theta}_k \cos(\theta_k-\theta_j) \right)
\end{align}

And the potential energy as:
\begin{align}
V &= g \sum_i m_i y_i \\
&= - g \sum_i m_i \sum_{j=1}^i \cos \theta_j
\end{align}

For convenience we set $m=1$

Which gives us:

\begin{align}
T &= \sum_i \left( \sum_{j=1}^i \frac{1}{2} \dot{\theta}_j^2 + \dot{\theta}_j \sum_{k=1}^{j-1}  \dot{\theta}_k \cos(\theta_k-\theta_j) \right) \\
&= \sum_{j=1}^n (n-j+1) \left( \frac{1}{2} \dot{\theta}_j^2 + \dot{\theta}_j \sum_{k=1}^{j-1}  \dot{\theta}_k \cos(\theta_k-\theta_j) \right)
\end{align}

\begin{align}
V &= - g \sum_i m_i \sum_{j=1}^i \cos \theta_j \\
&= - g \sum_i (n-i+1) \cos \theta_i
\end{align}

Hence the Lagrangian is in angular coordinates given as:
\begin{equation}
L = \sum_{j=1}^n (n-j+1) \left( \frac{1}{2} \dot{\theta}_j^2 + \dot{\theta}_j \sum_{k=1}^{j-1}  \dot{\theta}_k \cos(\theta_k-\theta_j) \right) + g \sum_j (n-j+1) \cos \theta_j
\end{equation}

We start evaluating terms needed for the Euler-Lagrange equation:

\begin{align}
\frac{\partial L}{\partial \dot{\theta}_i} &= \frac{\partial T}{\partial \dot{\theta}_i} \\
&= \frac{\partial }{\partial \dot{\theta}_i} \sum_{j=1}^n (n-j+1) \left( \frac{1}{2} \dot{\theta}_j^2 + \dot{\theta}_j \sum_{k=1}^{j-1}  \dot{\theta}_k \cos(\theta_k-\theta_j) \right) \\
&= (n-i+1) \dot{\theta}_i + \sum_{j=1}^n (n-j+1) \sum_{k=1}^{j-1} \left[\delta_{ij} \dot{\theta}_k + \dot{\theta}_j \delta_{ki} \right] \cos(\theta_k-\theta_j) \\
&= (n-i+1) \dot{\theta}_i + \sum_{j=1}^n (n-j+1) \sum_{k=1}^{j-1} \delta_{ij} \dot{\theta}_k \cos(\theta_k-\theta_j) + \dot{\theta}_j \delta_{ki} \cos(\theta_k-\theta_j) \\
&= (n-i+1) \dot{\theta}_i + \sum_{j=1}^n (n-j+1) \delta_{ij} \sum_{k=1}^{j-1} \dot{\theta}_k \cos(\theta_k-\theta_j) +  \sum_{j=1}^n (n-j+1) \sum_{k=1}^{j-1} \dot{\theta}_j \delta_{ki} \cos(\theta_k-\theta_j) \\
&= (n-i+1) \dot{\theta}_i + (n-i+1) \sum_{k=1}^{i-1} \dot{\theta}_k \cos(\theta_k-\theta_i) +  \sum_{j=i+1}^n (n-j+1) \dot{\theta}_j \cos(\theta_i-\theta_j) \\
&= (n-i+1) \dot{\theta}_i + (n-i+1) \sum_{j=1}^{i-1} \dot{\theta}_j \cos(\theta_j-\theta_i) +  \sum_{j=i+1}^n (n-j+1) \dot{\theta}_j \cos(\theta_i-\theta_j) \\
&= c(i) \dot{\theta}_i + c(i) \sum_{j=1}^{i-1} \dot{\theta}_j \cos(\theta_j-\theta_i) +  \sum_{j=i+1}^n c(i,j) \dot{\theta}_j \cos(\theta_i-\theta_j) \\
&= c(i) \dot{\theta}_i + \sum_{j \neq i}^{n} c(i,j) \dot{\theta}_j \cos(\theta_j-\theta_i)
\end{align}
where
\begin{equation}
c(i) = n - i +1
\end{equation}
and
\begin{equation}
c(i,j) = n - max(i,j) + 1
\end{equation}

Next we calculate the time derivative of it:
\begin{align}
\frac{d}{dt}\frac{\partial L}{\partial \dot{\theta}_i} &= \frac{d}{dt} \left( c(i) \dot{\theta}_i + \sum_{j \neq i}^{n} c(i,j) \dot{\theta}_j \cos(\theta_i-\theta_j) \right) \\
&= c(i) \ddot{\theta}_i + \sum_{j \neq i}^{n} c(i,j) \left[ \ddot{\theta}_j \cos(\theta_i-\theta_j) - \dot{\theta}_i \dot{\theta}_j \sin(\theta_i - \theta_j) + \dot{\theta}_j^2 \sin(\theta_i - \theta_j) \right] \\
&= \sum_{j=1}^{n} c(i,j) \left[ \ddot{\theta}_j \cos(\theta_i-\theta_j) - \dot{\theta}_i \dot{\theta}_j \sin(\theta_i - \theta_j) + \dot{\theta}_j^2 \sin(\theta_i - \theta_j) \right] \\
\end{align}

Next we calculate $\frac{\partial L}{\partial \theta_i}$:
\begin{align}
\frac{\partial L}{\partial \theta_i} &= \sum_{j=1}^n c(j) \dot{\theta}_j \sum_{k=1}^{j-1} \dot{\theta}_k (- \sin(\theta_k - \theta_j) \delta_{ki} + \sin(\theta_k-\theta_j)\delta_{ji} - g \sum_j c(j) \sin(\theta_j) \delta_{ij} \\
&= -\sum_{j=1}^n c(j) \dot{\theta}_j \sum_{k=1}^{j-1} \dot{\theta}_k \sin(\theta_k - \theta_j) \delta_{ki} + c(i) \dot{\theta}_i \sum_{k=1}^{i-1} \dot{\theta}_k \sin(\theta_k-\theta_i) - g c(i) \sin(\theta_i) \\
&= -\sum_{j=i+1}^n c(j) \dot{\theta}_j \dot{\theta}_i \sin(\theta_i - \theta_j) + c(i) \dot{\theta}_i \sum_{k=1}^{i-1} \dot{\theta}_k \sin(\theta_k-\theta_i) - g c(i) \sin(\theta_i) \\
&= -\sum_{j=1}^n c(i,j) \dot{\theta}_j \dot{\theta}_i \sin(\theta_i - \theta_j) - g c(i) \sin(\theta_i)
\end{align}

Hence for the Euler-Lagrange equation we get:
\begin{equation}
\sum_{j=1}^{n} c(i,j) \left[ \ddot{\theta}_j \cos(\theta_i-\theta_j) - \dot{\theta}_i \dot{\theta}_j \sin(\theta_i - \theta_j) + \dot{\theta}_j^2 \sin(\theta_i - \theta_j) \right] + \sum_{j=1}^n c(i,j) \dot{\theta}_j \dot{\theta}_i \sin(\theta_i - \theta_j) + g c(i) \sin(\theta_i) = 0
\end{equation}
Where some of the terms cancel out:
\begin{equation}
\sum_{j=1}^{n} c(i,j) \left[ \ddot{\theta}_j \cos(\theta_i-\theta_j) + \dot{\theta}_j^2 \sin(\theta_i - \theta_j) \right] + g c(i) \sin(\theta_i) = 0
\end{equation}
Which we can rearange as:
\begin{equation}
\sum_{j=1}^{n} c(i,j) \ddot{\theta}_j \cos(\theta_i-\theta_j) = - g c(i) \sin(\theta_i) - \sum_{j=1}^{n} c(i,j) \dot{\theta}_j^2 \sin(\theta_i - \theta_j) 
\end{equation}
Which we can solve as a linear system
\end{document}
